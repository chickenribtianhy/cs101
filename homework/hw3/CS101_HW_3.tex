\documentclass{article}

\usepackage{amsmath,amssymb,amsthm}
\usepackage{listings}
\usepackage{graphicx}
\usepackage[shortlabels]{enumitem}
\usepackage{tikz}
\usepackage[margin=1in]{geometry}
\usepackage{fancyhdr}
\usepackage{epsfig} %% for loading postscript figures
\usepackage{amsmath}
\usepackage{float}
\usepackage{amssymb}
\usepackage{caption}
\usepackage{subfigure}
\usepackage{graphics}
\usepackage{titlesec}
\usepackage{mathrsfs}
\usepackage{amsfonts}
\usepackage{indentfirst}
\usepackage{color}

\renewcommand{\baselinestretch}{1.2}%Adjust Line Spacing
%\geometry{left=2.0cm,right=2.0cm,top=2.0cm,bottom=2.0cm}% Adjust Margins of the File
\usepackage{tikz-qtree}
\usetikzlibrary{graphs}
\tikzset{every tree node/.style={minimum width=2em,draw,circle},
	blank/.style={draw=none},
	edge from parent/.style=
	{draw,edge from parent path={(\tikzparentnode) -- (\tikzchildnode)}},
	level distance=1.2cm}
\setlength{\parindent}{0pt}
%\setlength{\parskip}{5pt plus 1pt}
\setlength{\headheight}{13.6pt}
\newcommand\question[2]{\vspace{.25in}\hrule\textbf{#1: #2}\vspace{.5em}\hrule\vspace{.10in}}
\renewcommand\part[1]{\vspace{.10in}\textbf{(#1)}}
\newcommand\algorithm{\vspace{.10in}\textbf{Algorithm: }}
\newcommand\correctness{\vspace{.10in}\textbf{Correctness: }}
\newcommand\runtime{\vspace{.10in}\textbf{Running time: }}
\pagestyle{fancyplain}
% Create horizontal rule command with an argument of height
\newcommand{\horrule}[1]{\rule{\linewidth}{#1}}



% Set the title here
\title{
	\normalfont \normalsize
	\textsc{ShanghaiTech University} \\ [25pt]
	\horrule{0.5pt} \\[0.4cm] % Thin top horizontal rule
	\huge CS101 Algorithms and Data Structures\\ % The assignment title
	\LARGE Fall 2021\\
	\LARGE Homework 3\\
	\horrule{2pt} \\[0.5cm] % Thick bottom horizontal rule
}
% wrong usage of \author, never mind
\author{}
\date{Due date: 23:59, October 17, 2021}

% set the header and footer
\pagestyle{fancy}
\lhead{CS101 Algorithms and Data Structures}
\chead{Homework 3}
\rhead{Due date: 23:59, October 17, 2021}
\cfoot{\thepage}
\renewcommand{\headrulewidth}{0.4pt}
\newtheorem{Q}{Question}
% special settings for the first page
\fancypagestyle{firstpage}
{
	\renewcommand{\headrulewidth}{0pt}
	\fancyhf{}
	\fancyfoot[C]{\thepage}
}

% Add the support for auto numbering
% use \problem{title} or \problem[number]{title} to add a new problem
% also \subproblem is supported, just use it like \subsection
\newcounter{ProblemCounter}
\newcounter{oldvalue}
\newcommand{\problem}[2][-1]{
	\setcounter{oldvalue}{\value{secnumdepth}}
	\setcounter{secnumdepth}{0}
	\ifnum#1>-1
	\setcounter{ProblemCounter}{0}
	\else
	\stepcounter{ProblemCounter}
	\fi
	\section{Problem \arabic{ProblemCounter}: #2}
	\setcounter{secnumdepth}{\value{oldvalue}}
}
\newcommand{\subproblem}[1]{
	\setcounter{oldvalue}{\value{section}}
	\setcounter{section}{\value{ProblemCounter}}
	\subsection{#1}
	\setcounter{section}{\value{oldvalue}}
}

% \setmonofont{Consolas}
\definecolor{blve}{rgb}{0.3372549 , 0.61176471, 0.83921569}
\definecolor{gr33n}{rgb}{0.29019608, 0.7372549 , 0.64705882}
\makeatletter
\lst@InstallKeywords k{class}{classstyle}\slshape{classstyle}{}ld
\makeatother
\lstset{language=C++,
	basicstyle=\ttfamily,
	keywordstyle=\color{blve}\ttfamily,
	stringstyle=\color{red}\ttfamily,
	commentstyle=\color{magenta}\ttfamily,
	morecomment=[l][\color{magenta}]{\#},
	classstyle = \bfseries\color{gr33n}, 
	tabsize=4
}
\lstset{basicstyle=\ttfamily}
\begin{document}

\maketitle
\thispagestyle{firstpage}
%\newpage
\vspace{3ex}

\begin{enumerate}
	\item Please write your solutions in English.

	\item Submit your solutions to gradescope.com.

	\item Set your FULL NAME to your Chinese name and your STUDENT ID correctly in Account Settings.

	\item If you want to submit a handwritten version, scan it clearly. Camscanner is recommended.

	\item When submitting, match your solutions to the according problem numbers correctly.

	\item No late submission will be accepted.

	\item Violations to any of the above may result in zero grade.
\end{enumerate}
\newpage

\question{1}{(2'+2') Sorting Practice}
Given array:
$$1,   5,   3,   8,   7,   2,   4,   6$$
we want to sort this array in ascending order in-place.

Please \textbf{show your steps} taken by each type of sorting method on this array:
\begin{Q} \textbf{Quicksort.}
	After each partition during the algorithm, write the ordering of the list, \textbf{circle} the pivot that was used for that partition, and \textbf{underline} the sub-array being partitioned. Assume that the pivot is always the first item in the sublist being sorted.
\end{Q}
\textit{Example: \underline{\boxed{\textit{1}} 5 3 8 7 2 4 6}}\\
1 \underline{\boxed{5} 3 8 7 2 4 6}\\
1 \underline{\boxed{4} 3 2} 5 \underline{\boxed{7} 8 6}\\
1 \underline{\boxed{2} 3} 4 \underline{\boxed{6}} 7 \underline{\boxed{8}}\\
1 2 \underline{\boxed{3}} 4 5 6 7 8
\vspace{2cm}

\begin{Q}\textbf{Merge sort}. Show the intermediate merging steps. In each merging step, circle the subarray that is to be merged.
\end{Q}
\boxed{1} \boxed{5} \boxed{3} \boxed{8} \boxed{7} \boxed{2} \boxed{4} \boxed{6}\\
\boxed{1~5} \boxed{3~8} \boxed{2~7} \boxed{4~6}\\
\boxed{1~3~5~8} \boxed{2~4~6~7}\\
1 2 3 4 5 6 7 8

\pagebreak
%%%%%%%%%%%%%%%%%%%%%%%%%%%%%

\question{2}{(4$\times$1') Identifying Sorts}
Below you will find intermediate steps in performing various sorting algorithms that sort the input in ascending order on the same input list. The steps \textbf{do not necessarily represent consecutive steps in the algorithm} (that is, many steps are missing), but they are in the correct order. For each of them, \textbf{select the correct algorithm} \textbf{from the
	following choices}: insertion sort, bubble sort, quicksort (first element of sequence as pivot), and mergesort.
\\\\
\textbf{Input list}: 5, 6, 9, 4, 1, 2, 7, 10, 8, 0, 3


\begin{Q} \textcolor{white}{ }\\
	5, 6, 9, 1, 4, 2, 7, 10, 8, 0, 3\\
	5, 6, 1, 4, 9, 2, 7, 10, 0, 3, 8\\
	1, 4, 5, 6, 9, 0, 2, 3, 7, 8, 10\\
	\textbf{mergesort}
	\vspace{1cm}
\end{Q}

\begin{Q}\textcolor{white}{ }\\
	4, 1, 2, 0, 3, 5, 6, 9, 7, 10, 8\\
	1, 2, 0, 3, 4, 5, 6, 9, 7, 10, 8\\
	0, 1, 2, 3, 4, 5, 6, 7, 10, 8, 9\\
	\textbf{quicksort}
	\vspace{1cm}
\end{Q}

\begin{Q}\textcolor{white}{ }\\
	4, 5, 6, 9, 1, 2, 7, 10, 8, 0, 3\\
	1, 4, 5, 6, 9, 2, 7, 10, 8, 0, 3\\
	1, 2, 4, 5, 6, 9, 7, 10, 8, 0, 3\\
	\textbf{insertion sort}
	\vspace{1cm}
\end{Q}

\begin{Q}\textcolor{white}{ }\\
	5, 6, 4, 9, 1, 2, 7, 10, 8, 0, 3\\
	5, 6, 4, 1, 2, 7, 9, 8, 0, 3, 10\\
	5, 6, 1, 2, 4, 7, 9, 8, 0, 3, 10\\
	\textbf{bubble sort}

	\vspace{1cm}
\end{Q}

\pagebreak
%%%%%%%%%%%%%%%%%%%%%%%%%%%%%
\question{3}{(3$\times$1') Single Choice}
The following \textit{} questions are single choice questions, each question has \textbf{only one} correct answer. Select the correct answer.\\
\textit{Note: You should write those answers in the box below.}

\begin{table}[htbp]
	\begin{tabular}{|p{2cm}|p{2cm}|p{2cm}|}
		\hline
		Question 7 & Question 8 & Question 9 \\
		\hline
		A          & C          & C          \\
		\hline
	\end{tabular}
\end{table}


\begin{Q}
	Consider the quicksort algorithm which sorts elements in ascending order using the first element as pivot. Then which of the following input sequence will require a maximum number of comparisons when this algorithm is applied on it?
	\begin{enumerate}[(A)]
		\item 22 25 56 67 89
		\item 52 25 76 67 89
		\item 22 52 67 25 76
		\item 52 25 89 67 76
	\end{enumerate}
\end{Q}
\vspace{0.5cm}
\begin{Q}
	Which of the following statements is \textbf{NOT} true?
	\begin{enumerate}[(A)]
		\item The worst case time complexity of quicksort is $O(n^2)$.
		\item Given 2 sorted lists of size $m$ and $n$ respectively, and we want to merge them to one sorted list by mergesort. Then in the worst case, we need $m+n-1$ comparisons.
		\item The time complexity of quicksort, compared with mergesort, is less affected by the initial order of the input array.
		\item Comparing to quicksort, mergesort requires additional space complexity.
	\end{enumerate}
\end{Q}
\vspace{0.5cm}
\begin{Q} Given extra information about the input array, we may design sorting algorithms that perform faster than $O(N log N)$. Which of the following prior knowledge will lead to worst time complexity \textbf{slower} than $O(N)$?
	\begin{enumerate}[(A)]
		\item Knowing the input array has no more than $N$ inversions.
		\item Knowing the input array has exactly $(N^2-N)/2$ inversions.
		\item Knowing the input array has less than $N$ pairs of numbers that are not inversions.
		\item None of the above.
	\end{enumerate}
\end{Q}
\pagebreak


\begin{Q}\textbf{(4')Stable sort}\\
	We say a sort is ``stable'' if there is an implementation that it \textbf{always} preserves the original order of equal elements in an array without using additional space. For example, if we have an array: $4, 2, 3_a, 3_b, 1$, after sorting it has a chance to become $1, 2, 3_b, 3_a, 4$ ( $3_a$ and $3_b$ are equal when comparing), then this sorting algorithm is \textbf{NOT} stable.

	Among insertion sort, quicksort and mergesort, which of them are stable? (1') Briefly explain your answer. (3')
\end{Q}
\textbf{Solution}

Insertion sort and mergesort are stable.

As for insertion sort. If we define that every number should be inserted before the last number that is strictly greater than itself when probing from back to front,
then any pair of numbers that equal would be inserted in their original sequence, and be in the same sequence after their insertion respectively, since none of them can be greater than the other.

As for mergesort. In a mergesort we divide the whole array into pieces. When it comes to merging two subarrays, if we always take the number from the first subarray first when equal number
from both subarray are under considering, then we can keep the original sequence of the two equal numbers.

Quicksort is not stabe, because its uncertainty in choosing a pivot, and the swap process after each traverse.
\pagebreak

\end{document}